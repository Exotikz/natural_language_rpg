% !TeX root = ../main.tex
\chapter{Outils}

\section{Outils existants}
\paragraph{}
Si de prime abord tester peut être fait de manière local au code qui permet tout simplement s'assurer qu'une autre manière ne peut être utilisée, il n'y aucune garantie que ce soit vraiment le cas. Les tests vont de manière générale permettre de s'assurer qu'un comportement soit respecté, quel que soit le contexte.
Pour cela, il existe des outils permettant d'obtenir des tests dans de bonnes conditions. Les plus connus étant la suite "xUnit", comme CUnit pour le langage C, RUnit pour le langage R, JUnit pour le langage Java, etc.
Au sein même de Java, il y a deux principaux outils, JUnit et TestNG. Dans le cadre de ce projet, nous nous intéresserons à JUnit.
\paragraph{}
Afin de déterminer si toutes les fonctionnalités ont été testées (et surtout pour n'en oublier aucune), il est nécessaire de retenir tous les tests que l'on fait ainsi que de s'assurer qu'ils répondent bien au but. Cette tâche peut sembler longue et fastidieuse dès lors qu'un projet commence à prendre de l'ampleur. C'est pourquoi des outils comme JaCoCo (Java Code Coverage) permet de suivre l'avancement des tests effectués et de savoir si le test écrit a bien été exécuté.

\section{Outils utilisés}
\paragraph{\textbf{Maven}}
L'architecture du projet se base sur Maven, un outil Apache pour \textit{build} un projet (cf. \hyperref[Maven]{\color{blue}\underline{Maven}}).
\paragraph{JUnit}
Dans le cadre de ce stage sur le projet JUnit est utilisé pour tester les différentes classes. JUnit est soumis à la licence \textit{Eclipse Public License} (de \textit{Eclipse Foundation}) qui est \textit{open source}. Il s'agit de la version 4.11. La popularité de JUnit le rends accessible sur différent IDE (comme Eclipse, NetBeans ou encore Visual Studio Code, etc) et ainsi facilite son utilisation.

\paragraph{JaCoCo}
Toujours dans le même cadre, JaCoCo est utilisé de pair avec JUnit afin de s'apercevoir à quel point les tests ont été fructeux.
Lorsque l'on est sur une classe cela se présente sous forme de barre remplie en fonction du pourcentage de code couvert et lorsque l'on clique sur une méthode de cette classe, on obtient une vue sur le code directement avec en vert ce qui a été testé, en rouge ce qui ne l'est pas encore et en jaune qui signifie qu'il y a plusieurs branches à explorer.