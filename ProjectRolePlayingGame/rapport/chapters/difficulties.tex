% !TeX root = ../main.tex
\chapter{Difficultés}

\paragraph{}
La toute première difficulté rencontrée était le peu de cours que nous ayont eu sur les tests. Je ne savais pas vraiment comment m'y prendre dans un premier temps. C'est avec le premier package testé, \textbf{common}, que j'ai fait mes premières erreurs, comme par exemple faire plusieurs assert dans la même fonction de test. Et grâce aux retours de mon maître de stage, j'ai pu vite me rendre compte de comment procéder. Parmis mes erreurs on peut citer notamment :
\begin{itemize}
    \item Plusieurs assert dans la même fonction de test. C'est à éviter car si un assert échoue, les autres restant ne seront même pas exécutés.
    \item Ne pas mettre des valeurs significatives en constantes. Il faut éviter de codé "en dur".
\end{itemize}

\paragraph{}
Dans un second temps, j'avais trop de fonctions, pour par exemple tester si une valeur n'est pas nulle, ensuite travailler avec cette valeur et enfin tester une valeur qui ne serait pas égale. Cela me prenait trois fonctions à chaque fois. Or, j'ai découvert l'assomption de valeur ce qui m'a permis de réduire le nombre de méthode à une seule (cf. \hyperref[assume]{\color{blue}\underline{Test d'assomption}}).

\paragraph{}
Ensuite, j'ai eu beaucoup de mal pour par exemple passer dans certaine exception (comme celle de la méthode \textit{clone()} en java). C'est par les conseils de mon tuteur que j'en suis venu à faire des \textit{Mock} (ou encore \textit{Stub}), des "classes d'apparat" afin de pouvoir accéder à ce que l'on veut.

\paragraph{}
Par moment, la méthode que je testais ne faisait qu'échouer, et je ne savais pas pourquoi jusqu'à ce que je me rende compte que c'était un bug. Il est parfois difficile de savoir si le test est mal écrit ou bien si c'est un bug.

\paragraph{}
Le dernier package sur lequel j'ai travaillé, \textbf{gdt}, avait beaucoup de méthodes complexes et des tests d'intégrations. Relativement complexe à mettre en place, je n'y suis parvenu qu'avec l'aide de mon maître de stage.

\paragraph{}
Enfin, une des principales difficultés a été de comprendre le code que j'étais en train de tester. En effet, c'est un code écrit par plusieurs personnes, sur une longue période, et c'est une tâche complète à elle toute seule de comprendre comment fonctionne ce que je dois tester.