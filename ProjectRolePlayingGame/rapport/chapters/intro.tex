% !TeX root = ../main.tex
\chapter{Introduction}

\paragraph{}
Du 05 mars au 31 mai, j'ai effectué un stage au sein du GREYC situé au Campus 2 dans le bâtiment Sciences 3. Au cours de ce stage dans l'équipe MAD, j'ai pu m'intéresser aux tests unitaires sur un système multi-agent (référé par la suite dans ce rapport par \textbf{SMA} ou MAS (anglais pour Multi-Agent System)).
\paragraph{}
En programmation, plusieurs domaines sont nécessaires au développement d'un projet, allant de l'architecture au réseau, par conséquent un nombre important de lignes de codes peuvent être écrites. Pour s'assurer d'un bon fonctionnement de l'environnement, des tests doivent être effectués pour s'assurer du bon comportement de l'entité. Nous allons dans le cadre de ce projet nous intéresser tout particulièrement aux tests unitaires.
\paragraph{}
Si tester un programme peut paraître simple à première vue, lorsque celui-ci peut faire appel à des milliers de fonctions, cela peut vite devenir un casse-tête. C'est alors qu'intervient le test unitaire. Au lieu de tester si le programme tourne bien dans sa globalité, ce qu'on appelle les tests d'intégration, on va plutôt isoler les composants et les tester un à un séparément afin de s'assurer qu'ils fonctionnent exactement comme le développeur l'aurait souhaité indépendamment du reste.
\paragraph{}
Les tests unitaires, parfaitement taillés pour tester une fonctionnalité isolée, peuvent trouver leur limites lorsque ladite fonctionnalité doit faire appel à une autre fonctionnalité pour tourner pleinement. C'est à ce moment que les tests d'intégration font leur entrée en jeu, qui se concentrent sur un aspect plus large en fonctionnement que les test unitaires.
\paragraph{}
Dans ce stage, les test unitaires sont à effectuer sur un SMA pour lequel cela peut devenir très compliqué de tester certaine fonctionnalité individuellement.
Écrit dans le langage de programmation Java, les objectifs sont de tester de manière isolé ce qui est nécessaire à un SMA pour fonctionner grâce notamment au framework JUnit, et la couverture au possible du code en utilisant JaCoCo.

\info{JUnit est un framework pour Java. Il permet de réaliser des tests sur les classes.}
\info{JaCoCo est un outil qui permet de montrer la couverture totale du code réalisé sur un projet.}
%Un format classique est le suivant :
%(Optionnel) Un paragraphe pour exprimer les raisons de votre intérêt pour le sujet
%Un paragraphe sur le contexte technique général. Ex : “les objets connectés ouvrent de nombreuses nouvelles possibilités...”. Inclure des éléments factuels sur l’importance de ces technos (nombre d’utilisateurs, applications, bénéfice sociale/économique...). Une définition succincte des termes clefs.
%(Optionnel) Un paragraphe plus pour spécifier ce contexte : “dans ce cadre, on s’intéresse plus spécifiquement aux systèmes de tracking via GPS”. Inclure des éléments factuels sur l’importance de ces technos (nombre d’utilisateurs, applications, bénéfice sociale/économique...). Une définition succincte des termes clefs.

%Un paragraphe pour définir le problème ou les limites de l’existant. Ex : “Les objets permettant le tracking GPS limités par leur coût”. Y inclure les opportunités que résoudre ce problème peut apporter (tracking GPS bon marché).
%Un paragraphe sur la définition de l’objectif. C’est l’objectif spécifique sur lequel tout le reste du travail est basé, il faut être très précis dans les termes. Choisir un objectif SMART. Montrer que l’objectif résout bien le problème.
%Détailler les critères de qualité de cet objectif (ex : peu coûteux, sur batterie, déployable sur Android).

%