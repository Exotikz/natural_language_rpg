% !TeX root = ../main.tex
\chapter{Architecture}

\section{GDT}
\paragraph{}
Le GDT (Goal Decomposition Trees) est le projet sur lequel ce stage est centré et les tests unitaires sont à réaliser sur celui-ci. Le GDT est un modèle qui permet de pouvoir spécifier formellement le comportement des agents au sein d'un SMA, de construire des preuves sur l'exactitude de ces comportements et d'automatiquement générer une implémentation du comportement vérifié.

\section{Apache Maven}\label{Maven}
\paragraph{}
Pour citer Wikipédia :
\info{"[...] Maven est un outil de gestion et d'automatisation de production des projets logiciels Java en général et Java EE en particulier. Il est utilisé pour automatiser l'intégration continue lors d'un développement de logiciel. Maven est géré par l'organisation Apache Software Foundation."}
Plus spécifiquement ici, Maven permet d'intégrer facilement les frameworks comme JUnit, JaCoCo, JPL, etc. Cela simplifie de manière significative la façon de \textit{build} le projet, de l'\textit{update}, de lancer des tests, ...
Pour lancer les tests effectués jusqu'à présent, il faut lancer dans le terminal à la source du projet (à l'emplacement où se trouve le pom.xml, le fichier nécessaire à Maven) la commande suivante :

\section{org.JPL7}
\paragraph{}
Le module JPL ici utilisé sur ce projet permet de pouvoir transformer des objets Java en Prolog. Ceci est particulièrement utile car dans le cadre de ce projet, la construction de preuve est faite en Prolog, un langage de programmation logique pour lequel les théorèmes en sont la spécialité.

\section{Mon environnement}
\paragraph{}
Visual Studio Code est utilisé comme environnement de travail Java. Associé au package Java Test Runner.
Voici un script lancé avant chaque session :
Il me permet d'avoir les bons chemins (référés par \textit{path} ou encore \textit{classpath}) directement dans l'environnement de travail et d'ouvrir Visual Studio Code avec le projet "GDT - animator".