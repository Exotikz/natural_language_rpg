\documentclass[
    12pt,
	headsepline=on,
	footsepline=on,
	twoside=off,
	abstract=on,
	DIV=10
]{scrreprt}

\usepackage[utf8]{inputenc}
\usepackage{graphicx}
\usepackage[english, french]{babel}
\usepackage{multirow}
\usepackage[dvipsnames]{xcolor}
\usepackage[allbordercolors=white]{hyperref}
\usepackage{mdframed}
\usepackage{pgfplotstable}
\usepackage{tikz-3dplot}
\usepackage[OT1]{fontenc}
\usepackage{lipsum}
\usepackage{amsmath}
\usepackage{lscape} % permet de faire des pages en mode paysage
\usepackage{algorithmicx}
\usepackage[noend]{algpseudocode}
\usepackage{listings}
\usepackage{float}
\usepackage{hyperref}
\usepackage[sfdefault]{cabin}

\lstset{
    basicstyle=\ttfamily,
    breaklines=true,
    postbreak=\mbox{\textcolor{red}{$\hookrightarrow$}\space},
    language=Java,
    tabsize=4,
    numbers=left,
 	numberstyle=\tiny,
 	keywordstyle=\color{blue}\bfseries,
 	captionpos=b
}

\definecolor{link}{HTML}{4169E1}
\usepackage[bottom=2cm,footskip=8mm]{geometry}

\newmdenv[
    rightline=false,
    topline=false,
    bottomline=false,
    backgroundcolor=BurntOrange!5,
    fontcolor=BrickRed,
    linecolor=Red,
    linewidth=1pt
]{problemenv}

\newcommand{\problem}[1]{
\begin{problemenv}
\sffamily
#1
\end{problemenv}
}

\newmdenv[
    rightline=false,
    topline=false,
    bottomline=false,
    backgroundcolor=ForestGreen!5,
    fontcolor=OliveGreen,
    linecolor=Green,
    linewidth=1pt
]{resultenv}

\newcommand{\result}[1]{
\begin{resultenv}
\sffamily
#1
\end{resultenv}
}

\newmdenv[
rightline=false,
topline=false,
bottomline=false,
backgroundcolor=Cyan!5,
fontcolor=Blue,
linecolor=NavyBlue,
linewidth=1pt]{infoenv}

\newcommand{\info}[1]{
\begin{infoenv}
\sffamily
#1
\end{infoenv}
}

\newcommand{\img}[1]{
\begin{figure}[H]
	\centering
	\includegraphics[width=1\textwidth]{#1}	
\end{figure}
}

\newcommand{\imgwlegend}[2]{
	\begin{figure}[H]
		\centering
		\includegraphics[width=1\textwidth]{#1}	
		\caption{#2}
	\end{figure}
}

\newcommand{\imgwlegendwithwidth}[3]{
	\begin{figure}[H]
		\centering
		\includegraphics[width=#3\textwidth]{#1}	
		\caption{#2}
	\end{figure}
}

% Gestion d'abstracts multiples

\newenvironment{abstractpage}
{\cleardoublepage\vspace*{\fill}\thispagestyle{empty}}
{\vfill\cleardoublepage}

\renewenvironment{abstract}[1]
{\bigskip\selectlanguage{#1}%
	\begin{center}\bfseries\abstractname\end{center}}
{\par\bigskip}

% Gestion des keywords

\newcommand{\keywords}{\sffamily\textit{Keywords : }\bfseries}

%Page style

\pagestyle{headings}
\pagenumbering{arabic}


%Title page

\titlehead{
	\includegraphics[width=0.25\textwidth]{img/UNICAEN-logo-NOIR-horizontal.png}
	\hfill
	%\includegraphics[width=0.25\textwidth]{pics/}
}
\subject{
	\small
	Université de Caen Normandie\\
	UFR des Sciences\\
	Département Informatique\\
	\hfill\\
	1\up{ère} année de master
}
\title{
	\hrulefill
	%\hrulefill
	\vfill
	\Huge \bfseries COÉCRITURE EN LANGAGE NATUREL
}
\subtitle{
	Rapport de projet\\
	\hfill
	\\
	\hrulefill
	\hfill\\
}
\author{
	\small
	\includegraphics[ height=0.12\textheight, width=0.35\textwidth, keepaspectratio]{img/greyc}\\
	\hfill\\
	Jérôme LEGRIFFON, Anne WARTELLE
}
\date{}
\publishers{
	\small
	\begin{minipage}{0.6\textwidth}
		Tuteur : Loïs VANHÉE
	\end{minipage}
	\hfill
	\begin{minipage}{0.35\textwidth}
		Année 2019 - 2020\\
	\end{minipage}
}

\newcommand{\placeholderwarning}{
\problem{CECI EST UN PLACEHOLDER. À REMPLACER AVEC LES DONNÉES INDIQUÉES.}
}

\makeglossary
%redaction guide -> https://docs.google.com/document/d/1YfxGWD0GbRxs-OLxRxoA8Sg8OuVYTSKK8HX1ScFYlFA
\begin{document}

	\maketitle
	
	\pagenumbering{roman} 
	
	\tableofcontents
	
	\chapter*{Remerciements}
		\paragraph{} 
			Nous tenons à remercier notre tuteur de projet M. Loïs VANHÉE pour nous avoir formé et accompagné tout au long de cette expérience avec beaucoup de patience et de pédagogie, mon tuteur de stage M. Bruno ZANUTTINI pour ses conseils et explications, Christopher JACQUIOT pour son template \LaTeX.
			\clearpage
	
	\pagenumbering{arabic}

    % !TeX root = ../main.tex
\chapter{Introduction}

\paragraph{}
Du 05 mars au 31 mai, j'ai effectué un stage au sein du GREYC situé au Campus 2 dans le bâtiment Sciences 3. Au cours de ce stage dans l'équipe MAD, j'ai pu m'intéresser aux tests unitaires sur un système multi-agent (référé par la suite dans ce rapport par \textbf{SMA} ou MAS (anglais pour Multi-Agent System)).
\paragraph{}
En programmation, plusieurs domaines sont nécessaires au développement d'un projet, allant de l'architecture au réseau, par conséquent un nombre important de lignes de codes peuvent être écrites. Pour s'assurer d'un bon fonctionnement de l'environnement, des tests doivent être effectués pour s'assurer du bon comportement de l'entité. Nous allons dans le cadre de ce projet nous intéresser tout particulièrement aux tests unitaires.
\paragraph{}
Si tester un programme peut paraître simple à première vue, lorsque celui-ci peut faire appel à des milliers de fonctions, cela peut vite devenir un casse-tête. C'est alors qu'intervient le test unitaire. Au lieu de tester si le programme tourne bien dans sa globalité, ce qu'on appelle les tests d'intégration, on va plutôt isoler les composants et les tester un à un séparément afin de s'assurer qu'ils fonctionnent exactement comme le développeur l'aurait souhaité indépendamment du reste.
\paragraph{}
Les tests unitaires, parfaitement taillés pour tester une fonctionnalité isolée, peuvent trouver leur limites lorsque ladite fonctionnalité doit faire appel à une autre fonctionnalité pour tourner pleinement. C'est à ce moment que les tests d'intégration font leur entrée en jeu, qui se concentrent sur un aspect plus large en fonctionnement que les test unitaires.
\paragraph{}
Dans ce stage, les test unitaires sont à effectuer sur un SMA pour lequel cela peut devenir très compliqué de tester certaine fonctionnalité individuellement.
Écrit dans le langage de programmation Java, les objectifs sont de tester de manière isolé ce qui est nécessaire à un SMA pour fonctionner grâce notamment au framework JUnit, et la couverture au possible du code en utilisant JaCoCo.

\info{JUnit est un framework pour Java. Il permet de réaliser des tests sur les classes.}
\info{JaCoCo est un outil qui permet de montrer la couverture totale du code réalisé sur un projet.}
%Un format classique est le suivant :
%(Optionnel) Un paragraphe pour exprimer les raisons de votre intérêt pour le sujet
%Un paragraphe sur le contexte technique général. Ex : “les objets connectés ouvrent de nombreuses nouvelles possibilités...”. Inclure des éléments factuels sur l’importance de ces technos (nombre d’utilisateurs, applications, bénéfice sociale/économique...). Une définition succincte des termes clefs.
%(Optionnel) Un paragraphe plus pour spécifier ce contexte : “dans ce cadre, on s’intéresse plus spécifiquement aux systèmes de tracking via GPS”. Inclure des éléments factuels sur l’importance de ces technos (nombre d’utilisateurs, applications, bénéfice sociale/économique...). Une définition succincte des termes clefs.

%Un paragraphe pour définir le problème ou les limites de l’existant. Ex : “Les objets permettant le tracking GPS limités par leur coût”. Y inclure les opportunités que résoudre ce problème peut apporter (tracking GPS bon marché).
%Un paragraphe sur la définition de l’objectif. C’est l’objectif spécifique sur lequel tout le reste du travail est basé, il faut être très précis dans les termes. Choisir un objectif SMART. Montrer que l’objectif résout bien le problème.
%Détailler les critères de qualité de cet objectif (ex : peu coûteux, sur batterie, déployable sur Android).

%
    % !TeX root = ../main.tex
\chapter{Outils}

\section{Outils existants}
\paragraph{}
Si de prime abord tester peut être fait de manière local au code qui permet tout simplement s'assurer qu'une autre manière ne peut être utilisée, il n'y aucune garantie que ce soit vraiment le cas. Les tests vont de manière générale permettre de s'assurer qu'un comportement soit respecté, quel que soit le contexte.
Pour cela, il existe des outils permettant d'obtenir des tests dans de bonnes conditions. Les plus connus étant la suite "xUnit", comme CUnit pour le langage C, RUnit pour le langage R, JUnit pour le langage Java, etc.
Au sein même de Java, il y a deux principaux outils, JUnit et TestNG. Dans le cadre de ce projet, nous nous intéresserons à JUnit.
\paragraph{}
Afin de déterminer si toutes les fonctionnalités ont été testées (et surtout pour n'en oublier aucune), il est nécessaire de retenir tous les tests que l'on fait ainsi que de s'assurer qu'ils répondent bien au but. Cette tâche peut sembler longue et fastidieuse dès lors qu'un projet commence à prendre de l'ampleur. C'est pourquoi des outils comme JaCoCo (Java Code Coverage) permet de suivre l'avancement des tests effectués et de savoir si le test écrit a bien été exécuté.

\section{Outils utilisés}
\paragraph{\textbf{Maven}}
L'architecture du projet se base sur Maven, un outil Apache pour \textit{build} un projet (cf. \hyperref[Maven]{\color{blue}\underline{Maven}}).
\paragraph{JUnit}
Dans le cadre de ce stage sur le projet JUnit est utilisé pour tester les différentes classes. JUnit est soumis à la licence \textit{Eclipse Public License} (de \textit{Eclipse Foundation}) qui est \textit{open source}. Il s'agit de la version 4.11. La popularité de JUnit le rends accessible sur différent IDE (comme Eclipse, NetBeans ou encore Visual Studio Code, etc) et ainsi facilite son utilisation.

\paragraph{JaCoCo}
Toujours dans le même cadre, JaCoCo est utilisé de pair avec JUnit afin de s'apercevoir à quel point les tests ont été fructeux.
Lorsque l'on est sur une classe cela se présente sous forme de barre remplie en fonction du pourcentage de code couvert et lorsque l'on clique sur une méthode de cette classe, on obtient une vue sur le code directement avec en vert ce qui a été testé, en rouge ce qui ne l'est pas encore et en jaune qui signifie qu'il y a plusieurs branches à explorer.
    % !TeX root = ../main.tex
\chapter{Architecture}

\section{GDT}
\paragraph{}
Le GDT (Goal Decomposition Trees) est le projet sur lequel ce stage est centré et les tests unitaires sont à réaliser sur celui-ci. Le GDT est un modèle qui permet de pouvoir spécifier formellement le comportement des agents au sein d'un SMA, de construire des preuves sur l'exactitude de ces comportements et d'automatiquement générer une implémentation du comportement vérifié.

\section{Apache Maven}\label{Maven}
\paragraph{}
Pour citer Wikipédia :
\info{"[...] Maven est un outil de gestion et d'automatisation de production des projets logiciels Java en général et Java EE en particulier. Il est utilisé pour automatiser l'intégration continue lors d'un développement de logiciel. Maven est géré par l'organisation Apache Software Foundation."}
Plus spécifiquement ici, Maven permet d'intégrer facilement les frameworks comme JUnit, JaCoCo, JPL, etc. Cela simplifie de manière significative la façon de \textit{build} le projet, de l'\textit{update}, de lancer des tests, ...
Pour lancer les tests effectués jusqu'à présent, il faut lancer dans le terminal à la source du projet (à l'emplacement où se trouve le pom.xml, le fichier nécessaire à Maven) la commande suivante :

\section{org.JPL7}
\paragraph{}
Le module JPL ici utilisé sur ce projet permet de pouvoir transformer des objets Java en Prolog. Ceci est particulièrement utile car dans le cadre de ce projet, la construction de preuve est faite en Prolog, un langage de programmation logique pour lequel les théorèmes en sont la spécialité.

\section{Mon environnement}
\paragraph{}
Visual Studio Code est utilisé comme environnement de travail Java. Associé au package Java Test Runner.
Voici un script lancé avant chaque session :
Il me permet d'avoir les bons chemins (référés par \textit{path} ou encore \textit{classpath}) directement dans l'environnement de travail et d'ouvrir Visual Studio Code avec le projet "GDT - animator".
    % !TeX root = ../main.tex
\chapter{Réalisation technique}

\section{Initialisation d'un test}
\paragraph{}
Afin de réaliser les tests, il faut tout d'abord ajouter à l'architecture Maven un répertoire test en parallèle du main, c'est dans ce répertoire que seront tous les tests effectués.
Ensuite choisir une classe à tester et si cette classe se trouve par exemple dans le package common et donc le chemin est \textit{"src/main/java/common/Variable.java"}, alors le chemin du test donnera \textit{"src/test/java/common/TestVariable.java"} créer une classe que l'on nomme par convention \textbf{NomTest.java} ou \textbf{TestNom.java}.

\paragraph{}
Pour utiliser JUnit, il est nécessaire de l'importer et tout se trouve dans le package \textbf{org.junit}. Par exemple pour faire un simple test il faut importer Test du package: \textbf{\textit{import org.junit.Test;}}.

\section{Réaliser un test}
\paragraph{}
Dans le but de tester une classe, JUnit dispose de plusieurs façons de procéder avec de nombreux \textbf{assert} adaptés à différentes situations.
Le cas le plus courant lors de ce projet est simplement de s'assurer de l'égalité entre ce qu'on a et ce qui est attendu. Dans ce cas, un \textbf{assertEquals(\textit{valeur attendue, valeur courante})} est ce qu'il y a de plus simple à utiliser. Le test échouera si les deux valeurs ne correspondent pas.

\paragraph{}
Exemple d'assert de JUnit :
\begin{itemize}
    \item assertNull(\textit{objet}) / assertNotNull(\textit{objet}) : Le test réussi si l'objet est \textit{null} / \textit{non null}
    \item assertEquals(\textit{valeur attendue, valeur courante}) / assertNotEquals(\textit{valeur attendue, valeur courante}) : Le test réussi si les valeurs correspondent / diffèrent
    \item assertTrue(\textit{déclaration}) / assertFalse(\textit{déclaration}) : Le test réussi si la déclaration est vraie / fausse
\end{itemize}
De plus, il est possible d'utiliser des assomptions afin de permettre à un test d'être ignoré si la condition n'a pas été remplie au préalable. De ce fait, pour tester un objet pour lequel une de ses valeurs doit être vraie, on peut utiliser une méthode comme suit :\label{assume}
Ici par exemple, le test \textbf{assertEquals} sera ignoré si, avant tout, \textit{person} n'est pas un employé. C'est après s'être assuré que \textit{person} est un employé qu'on peut tester si son revenu est correct.

\paragraph{}
Il y a également une autre manière de tester une fonction, si on est sûr que celle-ci lève une exception. On doit écrire : \textbf{@Test(expected = NullPointerException.class)}, si par exemple NullPointerException va être levée. Dans le corps de la fonction il suffit juste d'appeler ce qui doit lever l'exception.

\paragraph{}
À savoir que pour tout les \textit{assert} et \textit{assume}, il est possible de mettre une chaîne de caractère en ajoutant un argument au début des paramètres afin de faire passer un message à travers le test : \textbf{assertEquals(\textit{"message", valeur attendue, valeur courante})}

\paragraph{Hamcrest}
En plus d'avoir JUnit et ses assert à disposition pour tester les différentes classes de java, il existe un autre outil qui vient s'ajouter à JUnit. Il s'agit d'Hamcrest. Cet outil permet d'écrire des tests de façon plus verbose afin qu'ils soient plus lisible et plus explicite. Ainsi, pour tester par exemple qu'une valeur est égale à 18, avec JUnit cela aurait donné :
\begin{itemize}
    \item assertEquals(majeur, 18);
\end{itemize}
Alors qu'avec Hamcrest cela peut donner :
\begin{itemize}
    \item assertThat(majeur, equalTo(18));
    \item assertThat(majeur, is(equalTo(18)));
    \item assertThat(majeur, is(18));
\end{itemize}
\info{Remarque : ces trois déclarations sont équivalentes}
On peut constater que la seconde manière de procéder, en utilisant par exemple le décorateur \textbf{\textit{is}}, augmente de manière significative la lisibilité des tests en transformant les déclarations de sorte que celles-ci ressemblent à des phrases.

\section{Cas concret de tests}
\paragraph{}
À priori, de manière générale, une classe de test par classe concrète est recommandée. De ce fait, il nous faut prendre en compte le problème de quoi, comment et pourquoi tester. Des classes comme les Types sont plutôt simple à tester individuellement mais des classes comme Variable sont, au contraire, plus difficile à tester car elles peuvent prendre plusieurs paramètres. C'est pourquoi elles nécessitent plus de tests, à priori, au moins un par type.
C'est à ce moment que rentre en jeu les tests de classes paramétrées. Ils permettent, d'une manière pratique et rapide de pouvoir construire de nombreuses instances d'un objet avec des paramètres personnalisés.

\paragraph{}
Pour simplement tester une classe qui n'a comme construction qu'un objet Type par exemple, il suffit simplement de l'initialiser une fois et ensuite d'entâmer la panoplie de test sur cette instance. Par exemple, une classe TypeEntier dans laquelle il n'y aurait qu'un int, et ses \textit{getters/setters}, devrait pouvoir être testée sans mettre un million de valeur, mais plutôt de mettre un entier et voir que ça fonctionne et de mettre un autre type pour constater que ça échoue.
\paragraph{}
Tandis que pour le cas dans lequel il y a plusieurs paramètres à prendre en compte, il est nécessaire d'avoir le plus de paramètres différents au possible afin d'être consistent. Voici un exemple de test avec classe paramétrées :
Dans ce cas précis, on peut facilement se rendre compte de ce qu'on attend de la classe. On met la valeur à tester et le retour attendu dans les paramètres et lors du test on s'assure que le comportement est bien respecté et que les paramètres concordent.

    % !TeX root = ../main.tex
\chapter{Difficultés}

\paragraph{}
La toute première difficulté rencontrée était le peu de cours que nous ayont eu sur les tests. Je ne savais pas vraiment comment m'y prendre dans un premier temps. C'est avec le premier package testé, \textbf{common}, que j'ai fait mes premières erreurs, comme par exemple faire plusieurs assert dans la même fonction de test. Et grâce aux retours de mon maître de stage, j'ai pu vite me rendre compte de comment procéder. Parmis mes erreurs on peut citer notamment :
\begin{itemize}
    \item Plusieurs assert dans la même fonction de test. C'est à éviter car si un assert échoue, les autres restant ne seront même pas exécutés.
    \item Ne pas mettre des valeurs significatives en constantes. Il faut éviter de codé "en dur".
\end{itemize}

\paragraph{}
Dans un second temps, j'avais trop de fonctions, pour par exemple tester si une valeur n'est pas nulle, ensuite travailler avec cette valeur et enfin tester une valeur qui ne serait pas égale. Cela me prenait trois fonctions à chaque fois. Or, j'ai découvert l'assomption de valeur ce qui m'a permis de réduire le nombre de méthode à une seule (cf. \hyperref[assume]{\color{blue}\underline{Test d'assomption}}).

\paragraph{}
Ensuite, j'ai eu beaucoup de mal pour par exemple passer dans certaine exception (comme celle de la méthode \textit{clone()} en java). C'est par les conseils de mon tuteur que j'en suis venu à faire des \textit{Mock} (ou encore \textit{Stub}), des "classes d'apparat" afin de pouvoir accéder à ce que l'on veut.

\paragraph{}
Par moment, la méthode que je testais ne faisait qu'échouer, et je ne savais pas pourquoi jusqu'à ce que je me rende compte que c'était un bug. Il est parfois difficile de savoir si le test est mal écrit ou bien si c'est un bug.

\paragraph{}
Le dernier package sur lequel j'ai travaillé, \textbf{gdt}, avait beaucoup de méthodes complexes et des tests d'intégrations. Relativement complexe à mettre en place, je n'y suis parvenu qu'avec l'aide de mon maître de stage.

\paragraph{}
Enfin, une des principales difficultés a été de comprendre le code que j'étais en train de tester. En effet, c'est un code écrit par plusieurs personnes, sur une longue période, et c'est une tâche complète à elle toute seule de comprendre comment fonctionne ce que je dois tester.
    % !TeX root = ../main.tex
\chapter{Perspectives}

\paragraph{}
Maintenant que je suis passé par tout ce que j'ai dû réaliser, je ferais les choses différemment dès le début et commencerais directement par utiliser les classes paramétrées plus souvent. Je prendrais plus de temps au début pour me renseigner plus profondément sur le projet plutôt que d'essayer de tester sans de réelles connaissances. Je pense que j'essayerais également de me renseigner sur une possible automation de certains tests de classes plus "primitives" (pour lesquelles peu de tests sont à exécuter).

\paragraph{}
Il reste encore des packages à tester, mais je pense que la suite fait plus partie des tests d'intégration pour le projet que de tests unitaires. Maintenant que j'ai une connaissance un peu plus développée du projet, je ferais plus de tests d'intégration afin de pouvoir couvrir encore plus de code.
    % !TeX root = ../main.tex
\chapter{Conclusion}
\paragraph{}
Lors de ce stage, l'objectif a été de faire des tests unitaires sur plusieurs packages du projet "GDT - animator", ils ont été réalisés à l'aide de JUnit et JaCoCo, avec un \textit{build} basé sur Apache Maven. Il m'a été donné d'apprendre énormément sur les SMA et plus généralement sur les tests dans un projet de grande ampleur avec des années de travaux.
J'ai dû faire preuve de patience face à l'échec, d'adaptation pour pouvoir avancer et savoir garder du recul dans les moments critiques afin de ne pas tomber dans le piège de la facilité. Ces 10 semaines de stage ont été une excellente expérience.
	\cleardoublepage
	\pagebreak
	
	\pagenumbering{Roman}
	
	\begin{thebibliography}{}
	    \bibitem{test-et-junit4}
	    Bruno MERMET
	    Tests et Junit 4
        \url{https://mermet.users.greyc.fr/Enseignement/CoursPDF/JUnit.pdf}
        
        \bibitem{git}
	    Bruno MERMET (2017)
	    Gestion de version avec GIT
        \url{https://mermet.users.greyc.fr/Enseignement/CoursPDF/gestionVersion-partie1.pdf}
        
        \bibitem{maven}
	    Bruno MERMET (novembre 2017)
	    Construction et gestion de développement avec Maven 3.0
        \url{https://mermet.users.greyc.fr/Enseignement/CoursPDF/maven.pdf}
        
        \bibitem{test-validation-logiciel}
	    Bruno MERMET (2010)
	    Le Test dans la validation du logiciel
        \url{https://mermet.users.greyc.fr/Enseignement/CoursPDF/testLogiciel.pdf}
        
        \bibitem{tutorials-point}
        tutorials point
        JUnit - Parameterized Test
        \url{https://www.tutorialspoint.com/junit/junit_parameterized_test.htm}
        
	\end{thebibliography}

%	\chapter{PARTIE}
	
%	\section{Nécessités}
	
%	\paragraph{}
	
%	\section{Problème}
	
%	\paragraph{}
	
%	\begin{problem}
	
%	\end{problem}
	
%	\section{Approches possibles}
	
%	\paragraph{}
	
%	\section{Approche utilisée}
	
%	\paragraph{}
	
%	\begin{result}
	
%	\end{result}
	
%	\section{Remarques sur les résultats obtenus}
	
%	\paragraph{}
	
%	\section{Pistes d'amélioration}
	
%	\paragraph{}
	
	
\end{document}